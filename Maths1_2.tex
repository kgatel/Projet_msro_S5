%% LyX 2.3.4.2 created this file.  For more info, see http://www.lyx.org/.
%% Do not edit unless you really know what you are doing.
\documentclass[english]{article}
\usepackage[T1]{fontenc}
\usepackage[latin9]{inputenc}
\usepackage{amsmath}
\usepackage{amssymb}
\usepackage{babel}
\begin{document}
1.$U\sim\mathbb{U}([0,1])$

$\mathbb{P}(U\leq t)=F_{U}(t)=\begin{cases}
0 & \text{si \ensuremath{t<0} }\\
\frac{t-0}{1-0}=t & \text{si \ensuremath{t}}\in[0,1]\\
1 & \text{si \ensuremath{t>1}}
\end{cases}$

$\mathbb{P}(G(U)<t)=\mathbb{P}(\frac{-1}{\lambda}\ln(1-U)<t)=\mathbb{P}(\ln(1-U)>-\lambda t)=\mathbb{P}(1-U>e^{-\lambda t})=\mathbb{P}(U<1-e^{-\lambda t})=F_{U}(1-e^{-\lambda t})=\begin{cases}
0 & \text{si \ensuremath{1-e^{-\lambda t}<0}}\\
1-e^{-\lambda t} & \text{si \ensuremath{\ensuremath{1}-}\ensuremath{e^{-\lambda t}}}\in[0,1]\\
1 & \text{si \ensuremath{1-e^{-\lambda t}>1}}
\end{cases}$

\[
1-e^{-\lambda t}<0\iff e^{-\lambda t}>1
\]
\[
\iff\underbrace{-\lambda}_{<0}t>0
\]
\[
\iff t<0
\]

\[
1-e^{-\lambda t}\in[0,1]\iff0\leq1-e^{-\lambda t}\leq1
\]
\[
\iff e^{-\lambda t}\leq1
\]
\[
\iff\underbrace{-\lambda}_{<0}t\leq0
\]
\[
\iff t\geq0
\]

\[
1-e^{-\lambda t}>1\iff e^{-\lambda t}<0\text{ aucun r�el t ne satisfait l'in�quation}
\]

D'o� $\mathbb{P}(G(U)<t)=\begin{cases}
0 & \text{si \ensuremath{t<0}}\\
1-e^{-\lambda t} & \text{si }t\geq0
\end{cases}$

Ainsi, G(U) suit bien une loi exponentielle de param�tre $\lambda$.

2. $\mathbb{P}(G(1-U)<t)=\mathbb{P}(\frac{-1}{\lambda}\ln(U)<t)=\mathbb{P}(\ln(U)>-\lambda t)=\mathbb{P}(U>e^{-\lambda t})=1-\mathbb{P}(U<e^{-\lambda t})=1-F_{U}(e^{-\lambda t})$

\[
F_{U}(e^{-\lambda t})=\begin{cases}
0 & \text{si \ensuremath{e^{-\lambda t}<0}}\text{ Aucun t ne satisfait l'in�quation }\\
e^{-\lambda t} & \text{si \ensuremath{e^{-\lambda t}}}\in[0,1]\\
1 & \text{si \ensuremath{e^{-\lambda t}>1}}
\end{cases}
\]

\[
-F_{U}(e^{-\lambda t})=\begin{cases}
-e^{-\lambda t} & \text{si \ensuremath{-}\ensuremath{e^{-\lambda t}}}\in[-1,0]\\
-1 & \text{si \ensuremath{-e^{-\lambda t}<-1}}
\end{cases}
\]
\[
\mathbb{P}(G(1-U)<t)=1-F_{U}(e^{-\lambda t})=\begin{cases}
1-e^{-\lambda t} & \text{si \ensuremath{1-}\ensuremath{e^{-\lambda t}}}\in[0,1]\iff t\geq0\\
0 & \text{si \ensuremath{1-e^{-\lambda t}<0}\ensuremath{\iff t<0}}
\end{cases}=F_{U}(1-e^{-\lambda t})
\]

On retombe bien sur la m�me loi exponentielle que G(U). Si $U\sim\mathbb{U}([0,1])$
on aura G(1-U)\textasciitilde G(U)\textasciitilde$exp(\lambda)$
\end{document}
