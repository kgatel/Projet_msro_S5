\documentclass[12,french]{report}
\usepackage{geometry}
\geometry{vmargin=3cm, hmargin=3cm}
\usepackage[T1]{fontenc}
\usepackage[utf8]{inputenc}
\usepackage[french]{babel}
\usepackage{graphicx}
\usepackage{amsmath}
\usepackage{amssymb}
\usepackage{sectsty}
\usepackage{authblk}
\usepackage{algpseudocode}
\usepackage{algorithm}
\usepackage{xspace}
\usepackage{mathtools}
\usepackage{mathrsfs}
\usepackage{enumitem}
\usepackage{titlesec}
\usepackage{hyperref}
\usepackage{xcolor}
\usepackage{caption}
\usepackage{float}
\usepackage{tabto}


%\AddThinSpaceBeforeFootnotes
%\FrenchFootnotes

\titleformat{\chapter}[hang]{\bf\Huge}{\thechapter.}{2pc}{}
\titlespacing*{\chapter}{10pt}{0pt}{40pt}[0pt]
\newcommand{\HRule}{\rule{\linewidth}{0.5mm}}

\providecommand{\keywords}[1]{\textbf{\textit{Keywords:}} #1}
\bibliographystyle{apalike}

\usepackage{hyperref}

\begin{document}
\hypersetup{pdfborder=0 0 0}

\begin{titlepage}

\begin{center}
	\vspace*{\stretch{1}}
	\textsc{{\LARGE Institut national des sciences appliquées de Rouen} \\ 			\vspace{6mm} {\Large INSA de Rouen}} \\
	\vspace{5mm}
	\includegraphics[width=0.4\textwidth]{./Images/insa}\\[1.0 cm]

	\textsc{\Large Projet Info GM3 - Vague 1 - Sujet 2}\\[0.6cm]

	% Title
	\HRule \\[0.5cm]
	{ \Huge \bfseries Simulation d'une loi exponentielle de paramètre $\lambda$}\\[0.2cm]
	\HRule \\[0.95cm]

	\includegraphics[width=0.6\textwidth]{./Images/Courbes_loi_exp}\\[0.9 cm]

	% Author and supervisor
	\begin{minipage}{0.4\textwidth}
		\begin{flushleft} \large
			\emph{Auteurs:}\\
			Thibaut \textsc{André-Gallis} \\
			{\small\href{mailto:thibaut.andregallis@insa-rouen.fr}{thibaut.andregallis@insa-rouen.fr}} \\
			Kévin \textsc{Gatel} \\
			{\small\href{mailto:kevin.gatel@insa-rouen.fr}{kevin.gatel@insa-				rouen.fr}}
		\end{flushleft}
	\end{minipage}
	\begin{minipage}{0.4\textwidth}
		\begin{flushright} \large
			\emph{Enseignant:} \\
			Ioana \textsc{Ciotir} \\
			{\small\href{mailto:ioana.ciotir@insa-rouen.fr}								{ioana.ciotir@insa-rouen.fr}}
		\end{flushright}
	\end{minipage}
	\vspace*{\stretch{1}}

	\vfill
	{\large 27 Novembre 2020}
\end{center}
\end{titlepage}

\tableofcontents

%\listoffigures

\renewcommand{\chaptername}{}
\chapter*{Introduction}
%\label{chapter:Introduction}
\addcontentsline{toc}{chapter}{Introduction}

La loi exponentielle modélise la durée de vie d'un phénomène sans mémoire ou sans vieillissement. En d'autres termes la probabilité que le phénomène dure au moins $h+t$ heures sachant qu'il a déjà duré $t$ heures sera la même que la probabilité de durer $h$ heures à partir de sa mise en fonction initiale. Elle peut notamment modéliser la vie des circuits électriques, résoudre des problématiques de durée de vie en général.\\

Dans ce projet nous allons nous intéresser à comment simuler et modéliser une loi exponentielle de paramètre $\lambda$. Dans un premier temps nous allons démontrer qu'il est possible d'obtenir une loi exponentielle en partant d'une loi uniforme, ensuite nous allons démontrer que prendre une loi uniforme $U$ ou bien $1-U$ ne change rien pour en obtenir une, enfin nous allons créer une loi exponentielle à expérimentalement à partir d'une loi uniforme à afin d'appliquer directement les parties précédentes.

\chapter{G(U) $\sim$ $\varepsilon(\lambda)$}

Soit $U$ une loi uniforme sur [0,1].

On a :

$$\mathbb{P}(U\leq t)=F_{U}(t)=\begin{cases}
0 & \text{si \ensuremath{t<0} }\\
\frac{t-0}{1-0}=t & \text{si \ensuremath{t}}\in[0,1]\\
1 & \text{si \ensuremath{t>1}}
\end{cases}$$

Soit $\lambda>0$ et $G$ une fonction telle que : \\

$$\begin{array}{ccccc}
	G & : & [0,1] & \longrightarrow & \mathbb{R}^{+} \\
	& & u & \longmapsto & \frac{-1}{\lambda}\ln(1-u) \\
\end{array}$$

Alors : \\
$$\mathbb{P}(G(U)<t)=\mathbb{P}(\frac{-1}{\lambda}\ln(1-U)<t)=\mathbb{P}(\ln(1-U)>-\lambda t)=\mathbb{P}(1-U>e^{-\lambda t})$$
Or :
$$\mathbb{P}(1-U>e^{-\lambda t})=\mathbb{P}(U<1-e^{-\lambda t})=F_{U}(1-e^{-\lambda t})=\begin{cases}
0 & \text{si \ensuremath{1-e^{-\lambda t}<0}}\\
1-e^{-\lambda t} & \text{si \ensuremath{\ensuremath{1}-}\ensuremath{e^{-\lambda t}}}\in[0,1]\\
1 & \text{si \ensuremath{1-e^{-\lambda t}>1}}
\end{cases}$$

On a :
$$\begin{array}{ccl}
	1-e^{-\lambda t}<0 & \iff & e^{-\lambda t}>1 \\
					   & \iff & \underbrace{-\lambda}_{<0}t>0 \\
					   & \iff & t<0 \\
\end{array}$$\\

Ensuite :
$$\begin{array}{ccl}
	1-e^{-\lambda t}\in[0,1] & \iff & 0\leq1-e^{-\lambda t}\leq1 \\
					   & \iff & e^{-\lambda t}\leq1 \\
					   & \iff & \underbrace{-\lambda}_{<0}t\leq0 \\
					   & \iff & t\geq0 \\
\end{array}$$\\

Enfin :
$$\begin{array}{ccl}
	1-e^{-\lambda t}>1 & \iff & e^{-\lambda t}<0 \\
	& \iff & t\in\emptyset \\
  \end{array}$$\\
  
Ainsi :
$$\mathbb{P}(G(U)<t)=\begin{cases}
0 & \text{si \ensuremath{t<0}}\\
1-e^{-\lambda t} & \text{si }t\geq0
\end{cases}$$\\

D'où $G(U) \sim \varepsilon(\lambda)$.
 
\chapter{G(U) $\sim$ G(1-U) $\sim$ $\varepsilon(\lambda)$}

On a :
$$\mathbb{P}(G(1-U)<t)=\mathbb{P}(\frac{-1}{\lambda}\ln(U)<t)=\mathbb{P}(\ln(U)>-\lambda t)$$\\
D'où : $$\mathbb{P}(\ln(U)>-\lambda t)=\mathbb{P}(U>e^{-\lambda t})=1-\mathbb{P}(U<e^{-\lambda t})=1-F_{U}(e^{-\lambda t})$$\\

Or : $$F_{U}(e^{-\lambda t})=\begin{cases}
0 & \text{si \ensuremath{e^{-\lambda t}<0}}\text{ Aucun t ne satisfait l'inéquation }\\
e^{-\lambda t} & \text{si \ensuremath{e^{-\lambda t}}}\in[0,1]\\
1 & \text{si \ensuremath{e^{-\lambda t}>1}}
\end{cases}$$\\
Donc :
$$-F_{U}(e^{-\lambda t})=\begin{cases}
-e^{-\lambda t} & \text{si \ensuremath{-}\ensuremath{e^{-\lambda t}}}\in[-1,0]\\
-1 & \text{si \ensuremath{-e^{-\lambda t}<-1}}
\end{cases}$$\\
Enfin :
$$1-F_{U}(e^{-\lambda t})=\mathbb{P}(G(1-U)<t)=\begin{cases}
1-e^{-\lambda t} & \text{si \ensuremath{1-}\ensuremath{e^{-\lambda t}}}\in[0,1]\iff t\geq0\\
0 & \text{si \ensuremath{1-e^{-\lambda t}<0}\ensuremath{\iff t<0}}
\end{cases}=F_{U}(1-e^{-\lambda t})$$\\

On retombe bien sur la même loi exponentielle que $G(U)$. \\

Pour conclure si $U\sim\mathbb{U}([0,1])$ on aura $G(1-U) \sim G(U)\sim \varepsilon(\lambda)$.

\chapter{Partie appliquée}

Afin d'illustrer par un exemple notre démonstration. Nous allons implémenter un programme qui simule un tirage aléatoire selon la loi de G(U), puis nous allons analyser les résultats obtenus pour voir si ils sont conforme avec la loi exponentielle. Nous avons rédigé notre programme en python qui possède une fonction aléatoire qui suit une loi uniforme sur [0,1]. Nous avons ensuite modifié les résultats retourné par cette fonction de la même façon que G(U) et rempli un tableau avec ces valeurs. Ensuite nous avons tracé la fonction de répartition d'une loi exponentielle de paramètre $\lambda$. Nous avons ensuite pour pouvoir exploiter nos résultats, utilisé un tableau de compteur qui comptabilise le nombre de valeur inférieur à une certaine constante dans le tableau contenant les valeurs tirées aléatoirement. En incrémentant cette constante on arrive à simuler une fonction de répartition de nos valeurs tirées aléatoirement. On trace ensuite les deux courbes pour pouvoir les comparer. Pour avoir d'autres éléments à étudier on compare également l'espérance théorique et expérimental. On calcule également numériquement l'écart moyen entre les deux courbes en plus du tracé de la courbe.
\begin{center}
	\includegraphics[width=1.5\textwidth]{./Images/Programme.png}\\[1.0 cm]
\end{center}


\end{document}
